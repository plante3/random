% A brainfuck interpreter implemented fully expandably in TeX.
% Usage: \bf <stdin> { <program> }

\catcode`@=11

\long\def\gobble#1{}
\long\def\gobbletwo#1#2{}
\long\def\gobblefour#1#2#3#4{}
\long\def\firstoftwo#1#2{#1}
\long\def\secondoftwo#1#2{#2}
\long\def\afterfi#1#2\fi{\fi#1}

\def\csarg#1#2{\expandafter#1\csname#2\endcsname}

\def\ifempty#1{\if\relax\detokenize{#1}\relax
	\expandafter\firstoftwo \else \expandafter\secondoftwo \fi}

\long\def\car#1#2\@nil{#1}
\long\def\zcar#1{%
	\ifempty{#1}0{\car#1\@nil}}

\long\def\cdr#1#2\@nil{#2}
\long\def\zcdr#1{%
	\ifempty{#1}0{\gobble#1}}

\long\def\rev#1{\revA{}#1\@nil}
\long\def\revA#1#2{\ifx\@nil#2\afterfi{#1}\else
	\afterfi{\revA{{#2}#1}}\fi}

\long\def\zlast#1{%
	\ifempty{#1}0{\expandafter\car \romannumeral-`0\rev{#1} \@nil}}

\long\def\zinit#1{%
	\ifempty{#1}0{\expandafter\rev\expandafter{%
		\romannumeral-`0\expandafter\gobble \romannumeral-`0\rev{#1}}}}

\def\hexdigit#1{\ifcase\numexpr#1\relax0\or 1\or 2\or 3\or 4%
	\or 5\or 6\or 7\or 8\or 9\or a\or b\or c\or d\or e\or f\fi}
\def\xstring{\ifnum\escapechar>\m@ne \expandafter\expandafter\expandafter\gobble \fi \string}
\begingroup
	\let\echar\empty
	\countdef\Ca=0 \countdef\Cb=1
	\Ca\z@ \Cb\z@
	\loop
		\edef\tmpa{\noexpand\xstring\xstring\\^\string^\hexdigit\Cb\hexdigit\Ca}%
		\scantokens\expandafter{\expandafter\edef\expandafter\tmpa\expandafter{\tmpa}}%
		\edef\echar{\unexpanded\expandafter{\echar}\noexpand\noexpand\noexpand\or\tmpa}%
		\advance\Ca\@ne
		\ifnum\Ca=\sixt@@n \Ca\z@ \advance\Cb\@ne \fi
	\ifnum\Cb<\sixt@@n\repeat

	\xdef\echar#1{\noexpand\ifcase\numexpr#1\relax \expandafter\gobbletwo\echar
		\noexpand\else\char#1 \noexpand\fi}
\endgroup

\def\bf#1#{\bf@{#1}}
\def\bf@#1#2{\bfA{}0{}{#1}#2\@nil}
\def\bfA#1#2#3#4#5{%
	% #1 bytes before; #2 current byte;
	% #3 bytes after;  #4 stdin;
	\ifx\@nil#5\expandafter\gobblefour \else
		\ifcsname bf:\unexpanded{#5}\endcsname
			\afterfi{\afterfi{%
				\csname bf:\unexpanded{#5}\endcsname}}%
		\else
			\expandafter\expandafter\expandafter\bfA
		\fi
	\fi {#1}{#2}{#3}{#4}}
\def\bfB#1#2#3#4{\expandafter\bfA\expanded{{#1}{#2}{#3}{#4}}}

\csarg\def{bf:>}#1#2#3#4{\bfB{#1{#2}}{\zcar{#3}}{\zcdr{#3}}{#4}}
\csarg\def{bf:<}#1#2#3#4{\bfB{\zinit{#1}}{\zlast{#1}}{{#2}#3}{#4}}
\csarg\def{bf:+}#1#2#3#4{\bfB{#1}{\the\numexpr#2+1}{#3}{#4}}
\csarg\def{bf:-}#1#2#3#4{\bfB{#1}{\the\numexpr#2-1}{#3}{#4}}
\csarg\def{bf:.}#1#2#3#4{\echar{#2}%
	\bfB{#1}{#2}{#3}{#4}}
\csarg\def{bf:,}#1#2#3#4{%
	\bfB{#1}{\ifempty{#4}0{\number\expandafter`\expanded{\zcar{#4}}}}{#3}{\ifempty{#4}{}{\gobble#4}}}

\csarg\def{bf:[}#1#2#3#4{%
	\bfI{{#1}{#2}{#3}{#4}}{}0}
\def\bfI#1#2#3#4{%
	\ifx[#4\expandafter\bfI\expanded{{#1}{#2[}{\the\numexpr#3+1}\expandafter}\else
	\ifx]#4\ifnum#3=\z@ \afterfi{\afterfi{\afterfi{\bfL#1{#2}}}}\else
		\afterfi{\afterfi{\afterfi{\expandafter\bfI\expanded{{#1}{#2]}{\the\numexpr#3-1}}}}}\fi
	\else \afterfi{\afterfi{\bfI{#1}{#2#4}{#3}}}\fi\fi}

\def\bfL#1#2#3#4#5{%
	\ifnum#2=\z@ \expandafter\secondoftwo \else \expandafter\firstoftwo \fi
	{\bfA{#1}{#2}{#3}{#4}#5[#5]}{\bfA{#1}{#2}{#3}{#4}}}

% Test programs

\edef\tmpa{\bf{++++++++++[>+++>++++>+++++>+++++++>+++++++++>++++++++++>+++++++++++>++++++++++++<<<<<<<<-]>>>>+++.<<<++.>>>>+++++++.>+++++++++.<<<<<.>>>>>>++++++++.<--------.>----.>+.<<<<<<<.>>>>>>----.<<.>>+++++++.<++.+.>-.>.<<<<<<<.>>+++++++++.<+.}}
\tt \meaning\tmpa % -> I am very naughty ;)

\edef\tmpb{\bf{++++++++[>++++[>++>+++>+++>+<<<<-]>+>+>->>+[<]<-]>>.>---.+++++++..+++.>>.<-.<.+++.------.--------.>>+.}}
\meaning\tmpb % -> Hello World!

\edef\tmpc{\bf abcde{,[+.,]}} 
\meaning\tmpc % -> bcdef

\bye
