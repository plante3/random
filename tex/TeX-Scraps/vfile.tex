% \usepackage{xcolor}
\catcode`\@=11

\def\sdef#1{\expandafter\def\csname#1\endcsname}
\def\slet#1{\expandafter\let\csname#1\endcsname}
\def\ssmap#1#2 {\ifx\;#2\expandafter\@gobbletwo \else #1{#2}\fi \ssmap#1}
\long\def\afterfi#1\fi{\fi#1}

\protected\def\@sanitizex{\count@\z@ \loop \catcode\count@12 \advance\count@\@ne \ifnum\count@<\@cclvi\repeat}

\newread\vfileR
\protected\def\vfile#1#2#3{%
    \relax\ifhmode\par\fi \smallskip
    \begingroup
    \immediate\openin\vfileR={\romannumeral-`0#1}%
        \ifeof\vfileR \immediate\closeout\vfileR \def\@sanitizex##1\endgroup{\engroup}\fi % eat everything
    \@sanitizex \endlinechar\m@ne
    \count@\@ne \count1\numexpr#2-1\relax \count3\numexpr#3+1\relax
    \everypar={\hangindent3em\hangafter\@ne}\raggedright
    \vfile@normal
        \loop \read\vfileR to\tmpa
            \ifnum\count@>\count1
                \noindent\llap{\vfile@lineno \the\count@ \quad}%
                \expandafter\vfile@
                \tmpa
                \vfile@
                \par
            \fi
            \advance\count@\@ne
            \ifnum\count@<\count3\repeat
    \immediate\closeout\vfileR
    \endgroup \par\smallskip
}
\def\vfile@#1{%
    \ifx\vfile@#1\expandafter\@gobble \else
    \allowbreak
    \ifcsname vfile/\number`#1\endcsname
        \csname vfile/\number`#1\expandafter\endcsname \fi
    #1%
    \fi \vfile@
}
\newbox\vfile@spbox
\newbox\vfile@tabbox
\sdef{vfile/\number`\ }#1{\copy\vfile@spbox}
\sdef{vfile/\number`\^^I}#1{\copy\vfile@tabbox}
\sdef{vfile/\number`\%}#1{\begingroup\def\par{\endgroup\par}\vfile@comment#1}
\sdef{vfile/\number`\\}#1\vfile@{\fi \vfile@csinit}
\def\vfile@csinit#1{%
    \ifx\vfile@#1\else
    \ifcase\uccode`#1%
        \@backslashchar#1%
        \expandafter\expandafter\expandafter\vfile@
    \else \afterfi\afterfi \vfile@cs#1\@@ \fi\fi
}
\def\vfile@cs#1\@@#2{%
    \ifx\vfile@#2%
        \vfile@cshl#1\@@
    \else
    \ifcase\uccode`#2%
        \vfile@cshl#1\@@
        \expandafter\expandafter\expandafter\vfile@\expandafter\expandafter\expandafter#2%
    \else \afterfi\afterfi \vfile@cs#1#2\@@ \fi\fi
}
\def\vfile@cshl#1\@@{%
    \begingroup
    \ifcsname vfile/#1\endcsname
        \csname vfile/#1\expandafter\endcsname
    \else
        \expandafter\ifpdfprimitive\csname#1\endcsname \vfile@csprimitive \fi
    \fi \@backslashchar#1%
    \endgroup
}

\def\vfile@cshlfam#1{\def\tmp{#1}\ssmap\vfile@@cshlfam}
\def\vfile@@cshlfam#1{\slet{vfile/#1}\tmp}

%% Example setup

\vfile@cshlfam{\color{blue}}%
m@ne z@ z@skip p@ @ne tw@ thr@@ sixt@@n @cclv @cclvi @m @M @Mi @Mii @Miii @Miv @MM maxdimen centering jot voidb@x \; %
\vfile@cshlfam{\color{red}}%
if ifcat ifnum ifdim ifodd ifvmode ifhmode ifmmode ifinner ifvoid ifhbox ifvbox ifx ifeof iftrue iffalse ifcase ifdefined ifcsname iffontchar ifpdfabsnum ifpdfabsdim ifincsname ifpdfprimitive unless else fi \; %

\def\vfile@normal     {\tt\color{black}}
\def\vfile@lineno     {\tt\scriptsize\color{gray}}
\def\vfile@comment    {\slshape\color{gray}}
\def\vfile@csprimitive{\color{magenta}}
\setbox\vfile@spbox \hbox{{\color{lightgray}\tt\char`\ }}
\setbox\vfile@tabbox\hbox to 4\wd\vfile@spbox{{\color{lightgray}%
    \hss
    $\vcenter{\pdfliteral{q .4 w 1 j -.8 -1 m -.8 1 l .8 0 l b Q}}\m@th$%
    \hss
}}%


%% Usage

% \uccode`@=1                % recognize @ as letter
% \vfile {foo.tex} {32} {70} % print lines 32,70 from foo.tex
